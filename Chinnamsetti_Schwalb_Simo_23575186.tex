% Options for packages loaded elsewhere
\PassOptionsToPackage{unicode}{hyperref}
\PassOptionsToPackage{hyphens}{url}
%
\documentclass[
]{article}
\usepackage{amsmath,amssymb}
\usepackage{iftex}
\ifPDFTeX
  \usepackage[T1]{fontenc}
  \usepackage[utf8]{inputenc}
  \usepackage{textcomp} % provide euro and other symbols
\else % if luatex or xetex
  \usepackage{unicode-math} % this also loads fontspec
  \defaultfontfeatures{Scale=MatchLowercase}
  \defaultfontfeatures[\rmfamily]{Ligatures=TeX,Scale=1}
\fi
\usepackage{lmodern}
\ifPDFTeX\else
  % xetex/luatex font selection
\fi
% Use upquote if available, for straight quotes in verbatim environments
\IfFileExists{upquote.sty}{\usepackage{upquote}}{}
\IfFileExists{microtype.sty}{% use microtype if available
  \usepackage[]{microtype}
  \UseMicrotypeSet[protrusion]{basicmath} % disable protrusion for tt fonts
}{}
\makeatletter
\@ifundefined{KOMAClassName}{% if non-KOMA class
  \IfFileExists{parskip.sty}{%
    \usepackage{parskip}
  }{% else
    \setlength{\parindent}{0pt}
    \setlength{\parskip}{6pt plus 2pt minus 1pt}}
}{% if KOMA class
  \KOMAoptions{parskip=half}}
\makeatother
\usepackage{xcolor}
\usepackage[margin=1in]{geometry}
\usepackage{color}
\usepackage{fancyvrb}
\newcommand{\VerbBar}{|}
\newcommand{\VERB}{\Verb[commandchars=\\\{\}]}
\DefineVerbatimEnvironment{Highlighting}{Verbatim}{commandchars=\\\{\}}
% Add ',fontsize=\small' for more characters per line
\usepackage{framed}
\definecolor{shadecolor}{RGB}{248,248,248}
\newenvironment{Shaded}{\begin{snugshade}}{\end{snugshade}}
\newcommand{\AlertTok}[1]{\textcolor[rgb]{0.94,0.16,0.16}{#1}}
\newcommand{\AnnotationTok}[1]{\textcolor[rgb]{0.56,0.35,0.01}{\textbf{\textit{#1}}}}
\newcommand{\AttributeTok}[1]{\textcolor[rgb]{0.13,0.29,0.53}{#1}}
\newcommand{\BaseNTok}[1]{\textcolor[rgb]{0.00,0.00,0.81}{#1}}
\newcommand{\BuiltInTok}[1]{#1}
\newcommand{\CharTok}[1]{\textcolor[rgb]{0.31,0.60,0.02}{#1}}
\newcommand{\CommentTok}[1]{\textcolor[rgb]{0.56,0.35,0.01}{\textit{#1}}}
\newcommand{\CommentVarTok}[1]{\textcolor[rgb]{0.56,0.35,0.01}{\textbf{\textit{#1}}}}
\newcommand{\ConstantTok}[1]{\textcolor[rgb]{0.56,0.35,0.01}{#1}}
\newcommand{\ControlFlowTok}[1]{\textcolor[rgb]{0.13,0.29,0.53}{\textbf{#1}}}
\newcommand{\DataTypeTok}[1]{\textcolor[rgb]{0.13,0.29,0.53}{#1}}
\newcommand{\DecValTok}[1]{\textcolor[rgb]{0.00,0.00,0.81}{#1}}
\newcommand{\DocumentationTok}[1]{\textcolor[rgb]{0.56,0.35,0.01}{\textbf{\textit{#1}}}}
\newcommand{\ErrorTok}[1]{\textcolor[rgb]{0.64,0.00,0.00}{\textbf{#1}}}
\newcommand{\ExtensionTok}[1]{#1}
\newcommand{\FloatTok}[1]{\textcolor[rgb]{0.00,0.00,0.81}{#1}}
\newcommand{\FunctionTok}[1]{\textcolor[rgb]{0.13,0.29,0.53}{\textbf{#1}}}
\newcommand{\ImportTok}[1]{#1}
\newcommand{\InformationTok}[1]{\textcolor[rgb]{0.56,0.35,0.01}{\textbf{\textit{#1}}}}
\newcommand{\KeywordTok}[1]{\textcolor[rgb]{0.13,0.29,0.53}{\textbf{#1}}}
\newcommand{\NormalTok}[1]{#1}
\newcommand{\OperatorTok}[1]{\textcolor[rgb]{0.81,0.36,0.00}{\textbf{#1}}}
\newcommand{\OtherTok}[1]{\textcolor[rgb]{0.56,0.35,0.01}{#1}}
\newcommand{\PreprocessorTok}[1]{\textcolor[rgb]{0.56,0.35,0.01}{\textit{#1}}}
\newcommand{\RegionMarkerTok}[1]{#1}
\newcommand{\SpecialCharTok}[1]{\textcolor[rgb]{0.81,0.36,0.00}{\textbf{#1}}}
\newcommand{\SpecialStringTok}[1]{\textcolor[rgb]{0.31,0.60,0.02}{#1}}
\newcommand{\StringTok}[1]{\textcolor[rgb]{0.31,0.60,0.02}{#1}}
\newcommand{\VariableTok}[1]{\textcolor[rgb]{0.00,0.00,0.00}{#1}}
\newcommand{\VerbatimStringTok}[1]{\textcolor[rgb]{0.31,0.60,0.02}{#1}}
\newcommand{\WarningTok}[1]{\textcolor[rgb]{0.56,0.35,0.01}{\textbf{\textit{#1}}}}
\usepackage{graphicx}
\makeatletter
\def\maxwidth{\ifdim\Gin@nat@width>\linewidth\linewidth\else\Gin@nat@width\fi}
\def\maxheight{\ifdim\Gin@nat@height>\textheight\textheight\else\Gin@nat@height\fi}
\makeatother
% Scale images if necessary, so that they will not overflow the page
% margins by default, and it is still possible to overwrite the defaults
% using explicit options in \includegraphics[width, height, ...]{}
\setkeys{Gin}{width=\maxwidth,height=\maxheight,keepaspectratio}
% Set default figure placement to htbp
\makeatletter
\def\fps@figure{htbp}
\makeatother
\setlength{\emergencystretch}{3em} % prevent overfull lines
\providecommand{\tightlist}{%
  \setlength{\itemsep}{0pt}\setlength{\parskip}{0pt}}
\setcounter{secnumdepth}{-\maxdimen} % remove section numbering
\ifLuaTeX
  \usepackage{selnolig}  % disable illegal ligatures
\fi
\usepackage{bookmark}
\IfFileExists{xurl.sty}{\usepackage{xurl}}{} % add URL line breaks if available
\urlstyle{same}
\hypersetup{
  pdftitle={Horse Racing Data Analysis Project - ML1 HS24},
  pdfauthor={sindhu.chinnamsetti@stud.hslu.ch, ronald.schwalb@stud.hslu.ch, eniko.simo@stud.hslu.ch},
  hidelinks,
  pdfcreator={LaTeX via pandoc}}

\title{Horse Racing Data Analysis Project - ML1 HS24}
\author{\href{mailto:sindhu.chinnamsetti@stud.hslu.ch}{\nolinkurl{sindhu.chinnamsetti@stud.hslu.ch}},
\href{mailto:ronald.schwalb@stud.hslu.ch}{\nolinkurl{ronald.schwalb@stud.hslu.ch}},
\href{mailto:eniko.simo@stud.hslu.ch}{\nolinkurl{eniko.simo@stud.hslu.ch}}}
\date{October 13th, 2024}

\begin{document}
\maketitle

\section{1) Data Cleaning}\label{data-cleaning}

{WIP}

\begin{Shaded}
\begin{Highlighting}[]
\CommentTok{\# Import the Data from CSV Files}
\NormalTok{df1 }\OtherTok{\textless{}{-}} \FunctionTok{read.csv}\NormalTok{(}\StringTok{"Lingfield\_AW\_2014\_2024\_flat.csv"}\NormalTok{)}
\NormalTok{df2 }\OtherTok{\textless{}{-}} \FunctionTok{read.csv}\NormalTok{(}\StringTok{"Lingfield\_AW\_2014\_2024\_jumps.csv"}\NormalTok{)}
\NormalTok{df3 }\OtherTok{\textless{}{-}} \FunctionTok{read.csv}\NormalTok{(}\StringTok{"Lingfield\_turf\_2014\_2024\_flat.csv"}\NormalTok{)}
\NormalTok{df4 }\OtherTok{\textless{}{-}} \FunctionTok{read.csv}\NormalTok{(}\StringTok{"Lingfield\_turf\_2014\_2024\_jumps.csv"}\NormalTok{)}

\NormalTok{df.horse }\OtherTok{\textless{}{-}} \FunctionTok{rbind}\NormalTok{(df1, df2, df3, df4)}
\end{Highlighting}
\end{Shaded}

\begin{itemize}
\tightlist
\item
  gotta diminish dataset size to:
\item
  max 10\^{}5 = 100,000 rows
\item
  10-20 predictors
\end{itemize}

Keep the following columns based on domain knowledge and careful
inspection of the dataset:

\begin{Shaded}
\begin{Highlighting}[]
\NormalTok{df.horse }\OtherTok{\textless{}{-}} \FunctionTok{subset}\NormalTok{(df.horse, }\AttributeTok{select=}\FunctionTok{c}\NormalTok{(date, race\_name,type, class, dist\_m, going, ran, pos, draw, horse, age, sex, lbs, hg, secs, jockey, trainer, prize))}
\end{Highlighting}
\end{Shaded}

Create a new column `won', assigning 1 if `pos' is 1, otherwise 0

\begin{Shaded}
\begin{Highlighting}[]
\NormalTok{df.horse}\SpecialCharTok{$}\NormalTok{won }\OtherTok{\textless{}{-}} \FunctionTok{ifelse}\NormalTok{(df.horse}\SpecialCharTok{$}\NormalTok{pos }\SpecialCharTok{==} \DecValTok{1}\NormalTok{, }\DecValTok{1}\NormalTok{, }\DecValTok{0}\NormalTok{)}
\CommentTok{\# Check the first few rows to ensure the new column has been added}
\FunctionTok{head}\NormalTok{(df.horse[}\FunctionTok{c}\NormalTok{(}\StringTok{"pos"}\NormalTok{,}\StringTok{"won"}\NormalTok{)],}\DecValTok{50}\NormalTok{)}
\end{Highlighting}
\end{Shaded}

\begin{verbatim}
##    pos won
## 1    1   1
## 2    2   0
## 3    3   0
## 4    4   0
## 5    5   0
## 6    6   0
## 7    7   0
## 8    8   0
## 9    9   0
## 10  10   0
## 11  11   0
## 12   1   1
## 13   2   0
## 14   3   0
## 15   4   0
## 16   5   0
## 17   6   0
## 18   7   0
## 19   8   0
## 20   1   1
## 21   2   0
## 22   3   0
## 23   4   0
## 24   5   0
## 25   6   0
## 26   7   0
## 27   8   0
## 28   9   0
## 29  10   0
## 30  11   0
## 31  12   0
## 32  13   0
## 33   1   1
## 34   2   0
## 35   3   0
## 36   4   0
## 37   5   0
## 38   6   0
## 39   7   0
## 40   8   0
## 41   9   0
## 42  10   0
## 43  11   0
## 44   1   1
## 45   2   0
## 46   3   0
## 47   4   0
## 48   5   0
## 49   6   0
## 50   7   0
\end{verbatim}

\begin{Shaded}
\begin{Highlighting}[]
\CommentTok{\# =\textgreater{} looks good}
\end{Highlighting}
\end{Shaded}

Check for missing values. NAs and empty strings

\begin{Shaded}
\begin{Highlighting}[]
\FunctionTok{colSums}\NormalTok{(}\FunctionTok{is.na}\NormalTok{(df.horse) }\SpecialCharTok{|}\NormalTok{ df.horse }\SpecialCharTok{==} \StringTok{""}\NormalTok{)}
\end{Highlighting}
\end{Shaded}

\begin{verbatim}
##      date race_name      type     class    dist_m     going       ran       pos 
##         0         0         0         0         0         0         0         0 
##      draw     horse       age       sex       lbs        hg      secs    jockey 
##      3950         0         0         0         0     34045         0         0 
##   trainer     prize       won 
##         0     22402         0
\end{verbatim}

\begin{Shaded}
\begin{Highlighting}[]
\CommentTok{\# let\textquotesingle{}s drop hg and prize because most of them are missing}
\CommentTok{\# but let\textquotesingle{}s keep draw because AFAIK it might influence outcome}
\CommentTok{\# keep it in mind that it only exists for Flat course type}

\NormalTok{df.horse }\OtherTok{\textless{}{-}} \FunctionTok{subset}\NormalTok{(df.horse, }\AttributeTok{select=}\SpecialCharTok{{-}}\FunctionTok{c}\NormalTok{(hg,prize))}
\end{Highlighting}
\end{Shaded}

Check datatypes.

\begin{Shaded}
\begin{Highlighting}[]
\FunctionTok{str}\NormalTok{(df.horse)}
\end{Highlighting}
\end{Shaded}

\begin{verbatim}
## 'data.frame':    53368 obs. of  17 variables:
##  $ date     : chr  "2014-01-04" "2014-01-04" "2014-01-04" "2014-01-04" ...
##  $ race_name: chr  "Coral Mobile Just Three Clicks To Bet Classified Claiming Stakes" "Coral Mobile Just Three Clicks To Bet Classified Claiming Stakes" "Coral Mobile Just Three Clicks To Bet Classified Claiming Stakes" "Coral Mobile Just Three Clicks To Bet Classified Claiming Stakes" ...
##  $ type     : chr  "Flat" "Flat" "Flat" "Flat" ...
##  $ class    : chr  "Class 6" "Class 6" "Class 6" "Class 6" ...
##  $ dist_m   : int  2012 2012 2012 2012 2012 2012 2012 2012 2012 2012 ...
##  $ going    : chr  "Standard" "Standard" "Standard" "Standard" ...
##  $ ran      : int  11 11 11 11 11 11 11 11 11 11 ...
##  $ pos      : chr  "1" "2" "3" "4" ...
##  $ draw     : int  7 12 4 3 5 11 6 1 2 9 ...
##  $ horse    : chr  "Ocean Applause (GB)" "Copperwood (GB)" "Paddys Saltantes (IRE)" "Exclusive Waters (IRE)" ...
##  $ age      : int  4 9 4 4 6 5 6 5 7 8 ...
##  $ sex      : chr  "G" "G" "G" "G" ...
##  $ lbs      : int  113 115 118 118 116 128 115 116 118 115 ...
##  $ secs     : chr  "124.96" "125.11" "125.41" "125.66" ...
##  $ jockey   : chr  "Joe Doyle" "Jimmy Quinn" "Luke Morris" "Andrea Atzeni" ...
##  $ trainer  : chr  "John Ryan" "Lee Carter" "J S Moore" "Gary Moore" ...
##  $ won      : num  1 0 0 0 0 0 0 0 0 0 ...
\end{verbatim}

Some variable types should be modified for better analysis.

Convert `date' column to date format

\begin{Shaded}
\begin{Highlighting}[]
\NormalTok{df.horse}\SpecialCharTok{$}\NormalTok{date }\OtherTok{\textless{}{-}} \FunctionTok{as.Date}\NormalTok{(}\FunctionTok{as.character}\NormalTok{(df.horse}\SpecialCharTok{$}\NormalTok{date), }\AttributeTok{format =} \StringTok{"\%Y{-}\%m{-}\%d"}\NormalTok{)}
\end{Highlighting}
\end{Shaded}

\begin{Shaded}
\begin{Highlighting}[]
\CommentTok{\# Convert all character columns to factors}
\NormalTok{df.horse[}\FunctionTok{sapply}\NormalTok{(df.horse, is.character)] }\OtherTok{\textless{}{-}} \FunctionTok{lapply}\NormalTok{(df.horse[}\FunctionTok{sapply}\NormalTok{(df.horse, is.character)], as.factor)}
\CommentTok{\# df.horse$race\_name \textless{}{-} as.character(df.horse$race\_name)}
\NormalTok{df.horse}\SpecialCharTok{$}\NormalTok{pos }\OtherTok{\textless{}{-}} \FunctionTok{as.integer}\NormalTok{(}\FunctionTok{as.character}\NormalTok{(df.horse}\SpecialCharTok{$}\NormalTok{pos))}
\end{Highlighting}
\end{Shaded}

\begin{verbatim}
## Warning: NAs introduced by coercion
\end{verbatim}

\begin{Shaded}
\begin{Highlighting}[]
\CommentTok{\# Convert secs to numeric after handling non{-}numeric values}
\NormalTok{df.horse}\SpecialCharTok{$}\NormalTok{secs }\OtherTok{\textless{}{-}} \FunctionTok{as.numeric}\NormalTok{(}\FunctionTok{as.character}\NormalTok{(df.horse}\SpecialCharTok{$}\NormalTok{secs))}
\end{Highlighting}
\end{Shaded}

\begin{verbatim}
## Warning: NAs introduced by coercion
\end{verbatim}

-\textgreater{} TOO MANY NAs INTRODUCED! -\textgreater{} COME BACK TO
THIS LATER!

Log-transform ``amounts'' -\textgreater{} at prof's recommendation In
our dataset, these variables can be considered amounts: - dist\_m - lbs
- secs

\begin{Shaded}
\begin{Highlighting}[]
\CommentTok{\# Log{-}transforming \textquotesingle{}dist\_m\textquotesingle{}, \textquotesingle{}lbs\textquotesingle{}, and \textquotesingle{}secs\textquotesingle{}}
\NormalTok{df.horse}\SpecialCharTok{$}\NormalTok{log\_dist\_m }\OtherTok{\textless{}{-}} \FunctionTok{log}\NormalTok{(df.horse}\SpecialCharTok{$}\NormalTok{dist\_m)}
\NormalTok{df.horse}\SpecialCharTok{$}\NormalTok{log\_lbs }\OtherTok{\textless{}{-}} \FunctionTok{log}\NormalTok{(df.horse}\SpecialCharTok{$}\NormalTok{lbs)}
\NormalTok{df.horse}\SpecialCharTok{$}\NormalTok{log\_secs }\OtherTok{\textless{}{-}} \FunctionTok{log}\NormalTok{(df.horse}\SpecialCharTok{$}\NormalTok{secs)}
\end{Highlighting}
\end{Shaded}

HOW TO VISUALIZE THE LOG AMOUNT TO COMPARE THEM WITH ORIGINAL??

Running the models take forever with our current dataset, so create a
randomized sample with 10\% of the data so that we can see results
faster at this early stage of the project.

\begin{Shaded}
\begin{Highlighting}[]
\CommentTok{\# Sample 10\% of the dataset for testing}
\NormalTok{df.horse\_sample }\OtherTok{\textless{}{-}}\NormalTok{ df.horse[}\FunctionTok{sample}\NormalTok{(}\FunctionTok{nrow}\NormalTok{(df.horse), }\FloatTok{0.1} \SpecialCharTok{*} \FunctionTok{nrow}\NormalTok{(df.horse)), ]}

\FunctionTok{dim}\NormalTok{(df.horse\_sample)}
\end{Highlighting}
\end{Shaded}

\begin{verbatim}
## [1] 5336   20
\end{verbatim}

\section{2) Linear Model}\label{linear-model}

Placeholder

\section{3) Generalised Linear Model -
Poisson}\label{generalised-linear-model---poisson}

Placeholder

\section{4) Generalised Linear Model -
Binomial}\label{generalised-linear-model---binomial}

WIP

\begin{Shaded}
\begin{Highlighting}[]
\NormalTok{glm.horse\_sample }\OtherTok{\textless{}{-}} \FunctionTok{glm}\NormalTok{(won }\SpecialCharTok{\textasciitilde{}}\NormalTok{ type }\SpecialCharTok{+}\NormalTok{ log\_dist\_m }\SpecialCharTok{+}\NormalTok{ going }\SpecialCharTok{+}\NormalTok{ age }\SpecialCharTok{+}\NormalTok{ sex }\SpecialCharTok{+}\NormalTok{ log\_lbs }\SpecialCharTok{+}\NormalTok{ log\_secs, }
                \AttributeTok{data =}\NormalTok{ df.horse\_sample,}
                \AttributeTok{family =} \StringTok{"binomial"}\NormalTok{)}
\FunctionTok{summary}\NormalTok{(glm.horse\_sample)}
\end{Highlighting}
\end{Shaded}

\begin{verbatim}
## 
## Call:
## glm(formula = won ~ type + log_dist_m + going + age + sex + log_lbs + 
##     log_secs, family = "binomial", data = df.horse_sample)
## 
## Coefficients:
##                         Estimate Std. Error z value Pr(>|z|)    
## (Intercept)           -1.538e+02  1.011e+01 -15.214  < 2e-16 ***
## typeFlat              -4.125e+00  6.163e-01  -6.694 2.18e-11 ***
## typeHurdle            -9.920e-01  4.186e-01  -2.370  0.01781 *  
## typeNH Flat           -3.909e+00  6.543e-01  -5.974 2.32e-09 ***
## log_dist_m             4.000e+01  2.745e+00  14.576  < 2e-16 ***
## goingGood To Firm      1.680e-01  2.890e-01   0.581  0.56095    
## goingGood To Soft      7.587e-01  3.969e-01   1.912  0.05592 .  
## goingHeavy             3.404e+00  5.041e-01   6.752 1.45e-11 ***
## goingSoft              1.984e+00  3.809e-01   5.208 1.90e-07 ***
## goingStandard          2.538e-03  2.494e-01   0.010  0.99188    
## goingStandard To Slow -1.149e-02  3.201e-01  -0.036  0.97138    
## age                   -9.069e-02  3.110e-02  -2.916  0.00354 ** 
## sexF                  -2.126e-01  1.581e-01  -1.345  0.17865    
## sexG                  -3.192e-01  1.567e-01  -2.037  0.04162 *  
## sexH                  -5.706e-01  5.206e-01  -1.096  0.27302    
## sexM                  -7.668e-01  2.576e-01  -2.976  0.00292 ** 
## log_lbs                5.720e+00  1.056e+00   5.417 6.05e-08 ***
## log_secs              -3.631e+01  2.496e+00 -14.545  < 2e-16 ***
## ---
## Signif. codes:  0 '***' 0.001 '**' 0.01 '*' 0.05 '.' 0.1 ' ' 1
## 
## (Dispersion parameter for binomial family taken to be 1)
## 
##     Null deviance: 3805.6  on 5269  degrees of freedom
## Residual deviance: 3466.8  on 5252  degrees of freedom
##   (66 observations deleted due to missingness)
## AIC: 3502.8
## 
## Number of Fisher Scoring iterations: 5
\end{verbatim}

\textless span style=``color:red'';\textgreater HOW TO INTERPRET ALL
THESE FACTOR VARIABLES? SHOULD THEY ALL BE FACTORS OR IS THERE A BETTER
WAY? I DON'T REMEMBER SO MANY FACTORS FROM THE LECTURES.

\section{5) Generalised Additive
Model}\label{generalised-additive-model}

Placeholder

\section{6) Neural Network}\label{neural-network}

Placeholder

\section{7) Support Vector Machine}\label{support-vector-machine}

Placeholder

\section{8) Use of Generative AI}\label{use-of-generative-ai}

Placeholder

\section{9) Conclusion}\label{conclusion}

Placeholder

\end{document}
